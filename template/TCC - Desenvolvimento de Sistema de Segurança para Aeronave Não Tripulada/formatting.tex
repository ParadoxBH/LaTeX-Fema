
%permite pegar a data da build
\usepackage{datetime}

% torna a fonte arial e justificada
\usepackage{helvet}
\renewcommand{\familydefault}{\sfdefault}
\usepackage[T1]{fontenc}

%insere 1,15 de distancia entre paragrafos
\usepackage{setspace}
\setstretch{1.15}

%remove o recuo de paragrafo
\setlength{\parindent}{0pt}

% função utilizada para criar recuo em paragrafo
\usepackage{changepage}

% função subtrair
\def\subtraiNumeros#1#2{%
	\number\numexpr#1-#2\relax%
}


% biblioteca de glossario
\usepackage[nonumberlist, acronym, toc]{glossaries}
\usepackage{glossaries-extra}
\renewcommand*{\glossarysection}[2][]{}
\makenoidxglossaries
\newglossaryentry{VANTs}
{
	name={VANT},
	plural={VANTs},
	description={Veículo Aéreo Não Tripulado}
}
\newglossaryentry{ANAC}
{
	name={ANAC},
	description={Agência Nacional de Aviação Civil}
}
\newglossaryentry{RPA}
{
	name={RPA},
	description={Aeronave Remotamente Pilotada}
}
\newglossaryentry{DECEA}
{
	name={DECEA},
	description={Departamento de Controle do Espaço Aéreo}
}
\newglossaryentry{ANATEL}
{
	name={ANATEL},
	description={Agência Nacional de Telecomunicações}
}
\newglossaryentry{IBAPE}
{
	name={IBAPE},
	description={Instituto Brasileiro de Avaliações e Perícias de Engenharia}
}
\newglossaryentry{ESC}
{
	name={ESC},
	plural={ESCs},
	description={Controlador Eletrônico de Velocidade}
}
\newglossaryentry{RPM}
{
	name={RPM},
	description={Rotação por Minuto}
}
\newglossaryentry{GPS}
{
	name={GPS},
	description={Sistema de Posicionamento Global}
}
\newglossaryentry{LIDAR}
{
	name={LIDAR},
	description={Detecção de luz e alcance}
}
\newglossaryentry{IMU}
{
	name={IMU},
	description={Unidade de Medição Inercial}
}
\newglossaryentry{EKF}
{
	name={EKF},
	description={Filtro Kalman Extendido}
}
\newglossaryentry{AVGAS}
{
	name={AVGAS},
	description={Gasolina de Aviação}
}
\newglossaryentry{PMD}
{
	name={PMD},
	description={Peso Máximo de Decolagem}
}
\newglossaryentry{SISANT}
{
	name={SISANT},
	description={Sistema de Aeronaves não Tripuladas}
}
\newglossaryentry{CHT}
{
	name={CHT},
	description={Certificado de Habilitação Técnica}
}
\newglossaryentry{CA}
{
name={CA},
description={Certificado de Aeronavegabilidade}
}
\newglossaryentry{CAARF}
{
name={CAARF},
description={Certificado de Aeronavegabilidade para Aeronaves Recém-Fabricadas}
}
\newglossaryentry{VTOLs}
{
name={VTOL},
plural={VTOLs},
description={Certificado de Aeronavegabilidade para Aeronaves Recém-Fabricadas}
}
\def\printpageglossary{
	\newpage\section*{\protect\centering{GLOSSÁRIO}}
	\printnoidxglossary[title={GLOSSÁRIO},nonumberlist]
}


% permite inserir imagens no documento
\usepackage{graphicx}

% permite inserir imagens no documento
\usepackage{tocloft}

% biblioteca de referencias de imagem
\usepackage{tocloft}

% biblioteca pra usar \ifstrequal
\usepackage{etoolbox}

% colocando borda nas imagens
\fboxsep=2pt %distancia da borda das imagens
\fboxrule=2pt %largura da borda das imagens
\usepackage{xcolor}% importa o pacote de cores
\def\colorBorder{black!10} %cor da borda das imagens
\def\colorPadding{white} %cor do fundo das imagens

\usepackage{microtype} % melhora a disposição do texto nas linhas
\usepackage{enumitem} % para usar a biblioteca de inumeração de capitulo

% bibliotecas do glossario
\usepackage{hyperref} % cria ancora em todos as referencias
\usepackage{glossaries}



%============================================================================
%funções de norma IMESA
%============================================================================

%citação exibir somente sobrenome e ano


% Definir um novo estilo enumerate com o texto "Capítulo" antes da numeração
\newlist{myenum}{enumerate}{1}
\setlist[myenum]{label=\textbf{Capítulo \arabic*,},leftmargin=*}
\newenvironment{chapterItem}[1]
{\chapter{#1}
	\begin{myenum}}
	{\end{myenum}}

% sinalizar caixa de aviso do orientador
\def\aviso#1{
	\p{\colorbox{red!25}{\parbox{\linewidth}{
				\textbf{#1}}}}
}

%alterando as margem da pagina
\usepackage[left=2cm,right=2cm,top=3cm,bottom=2cm]{geometry}

%função para pular linha
\def\jumpLine#1{
	\vspace{12pt}
	\ifnum#1>1
	\jumpLine{\subtraiNumeros{#1}{1}}
	\fi
}

% formatando sumario de imagens
\renewcommand{\listfigurename}{\begin{center}LISTA DE ILUSTRAÇÕES\end{center}}% altera o titulo da lista de figuras de List of figures para LISTA DE ILUSTRAÇÕES
\renewcommand{\cftfigpresnum}{\textbf{Figura\ }}%insere o texto Imagem antes do numero e torna negrito o texto
\renewcommand{\cftfigaftersnum}{:}%insere : apos o numero da imagem
\setlength{\cftfignumwidth}{2.2cm}%adiciona uma espaço antes do nome da imagem para caber a numeração + texto na esquerda do titulo

% remove o texto Reference no comando \bibliography
\renewcommand{\refname}{\vspace{-1.5cm}} 

% realiza a formatação da pagina de Sumario
\renewcommand{\contentsname}{\begin{center}SUMÁRIO\end{center}}
\usepackage{tocloft}
\usepackage{titlesec}
\renewcommand{\cftdotsep}{1} % Define a separação entre os pontos
\renewcommand{\cftsecleader}{\cftdotfill{\cftdotsep}} % Preenchimento de pontos até a página
\def\spacesumario{3em}%Define o espaço igualmente este os itens presente no sumario

\setlength{\cftbeforesecskip}{0pt}% Remove o espaço entre seções
\cftsetindents{section}{0pt}{\spacesumario}% Define o espaço a esquerda das seções
\renewcommand{\cftsecfont}{\fontsize{13pt}{13pt}\bfseries} % Formata o texto da seções

\cftsetindents{subsection}{0pt}{\spacesumario}% Define o espaço a esquerda das subseções
\renewcommand{\cftsubsecfont}{\fontsize{12pt}{12pt}\normalfont} % Formata o texto da subseções

\cftsetindents{subsubsection}{0pt}{\spacesumario}% Define o espaço a esquerda do item das subseções
\renewcommand{\cftsubsubsecfont}{\fontsize{11pt}{11pt}\bfseries} % Formata o texto do item da subseção


%função para inserir uma citação direta
\def\citacaoDireta#1#2{
	\par Segundo \citeonline{#1}:\textoDireta{3.5cm}{#2}
}

%função para inserir texto deslocado a direita
\def\textoDireta#1#2{
	\jumpLine{2}\begin{adjustwidth}{#1}{0cm} #2\end{adjustwidth}\jumpLine{2}
}

%referencia um script no artigo, deixa o texto dentro de uma caixa cinza clara
\definecolor{ligthgray}{RGB}{225,225,225}
\def\refcode#1{
\colorbox{ligthgray}{#1}
}

%função paragrafo
\def\p#1{\par #1 \vspace{10pt}\par}

%função formatação de item
\def\showItem#1#2{\item\textbf{#1: }#2}

%altera a forma que é exibido a legenda das imagem
\usepackage{subcaption}
\DeclareCaptionFormat{custom}% cria uma formatação customizada para as legendas, tornando maiusculo o texto Figure X:
{%
	\textbf{#1#2}#3
}
\captionsetup{format=custom}% aplica a formatação customizada criada acima
\captionsetup[figure]{name=Figura}% altera da legenda das imagem de Figure X: para Figura X:

%função verifica se o author citado é voce ou alguma referencia % use {me} nas referencia para citar si mesmo
\def\showauthor#1{
	\ifstrequal{#1}{me}
	{O Autor}
	{\citeonline{#1}}
}

%função inserir imagem no artigo % \citeimage{author}{.png}{descricao}
\usepackage{caption}
\def\insertimage#1#2#3{
	\begin{figure}[htbp]
		\centering
		\fcolorbox{\colorBorder}{\colorPadding}{}
		\caption{#3}
		\textbf{Fonte: }\showauthor{#1}
		\label{fig:#2}
	\end{figure}
}
%função inserir imagem no artigo % \citeimage{author}{.png}{descricao}{tamanho}
\usepackage{caption}
\def\insertimagesize#1#2#3#4{
\begin{figure}[htbp]
	\centering
	\fcolorbox{\colorBorder}{\colorPadding}{}
	\caption{#3}
	\textbf{Fonte: }\showauthor{#1}
	\label{fig:#2}
\end{figure}
}

%função: citar imagem inserida no artigo
\def\citimage#1{
	\ref{fig:#1}
}


% cria uma linha dedicada a assinatura \signature{função}{nome}
\def\signature#1#2{
	\jumpLine{1}
	\begin{tabular}{p{3cm}@{}p{12cm}}
		\textbf{#1 :} & \hrulefill \\
		&\centering{#2} \\
	\end{tabular}
}

%cria função para gerar pagina de referencias 
\usepackage[alf,abnt-emphasize=bf]{abntex2cite}

\def\printreferences{
\newpage\section*{REFERÊNCIAS} \addcontentsline{toc}{section}{REFERÊNCIAS}
\bibliography{referencias.bib}
}
\usepackage{babel}


%formata o glossario do documento
\addto\captionsportuguese{
	\renewcommand{\glossaryname}{\centering{GLOSSÁRIO}}
}

%============================================================================
% paginas
%============================================================================

\usepackage{afterpage} % permite criar paginaspersonalizadas

% criando variaveis de ambiente da capa
\newcommand{\orientando}[1]{\renewcommand{\orientando}{#1}}
\newcommand{\titulo}[1]{\renewcommand{\titulo}{#1}}
\newcommand{\orientador}[1]{\renewcommand{\orientador}{#1}}
\newcommand{\examinador}[1]{\renewcommand{\examinador}{#1}}

	
% desenhando capa
\usepackage{titling}
\newcommand{\showcapa}{
	\begin{titlepage}
		\centering
		\includegraphics[keepaspectratio]{Imagens/fema}
		\jumpLine{7}
		\textbf{\orientando}
		\vfill
		\textbf{\titulo}
		\vfill
		\textbf{Assis/SP\\\the\year}
	\end{titlepage}
	\newpage\paginaApresentacao
	\newpage\paginaComissaoExaminadora
}

% desenhando apresentacao
\def\paginaApresentacao{
	\begin{titlepage}
		\centering
		\includegraphics[keepaspectratio]{Imagens/fema}
		\jumpLine{7}
		\textbf{\orientando}
		\vfill
		\textbf{\titulo}
		
		\textoDireta{8cm}{\jumpLine{2}\footnotesize{
			\p{Trabalho de Conclusão de Curso apresentado ao curso de Ciência da Computação do Instituto Municipal de Ensino Superior de Assis – IMESA e a Fundação Educacional do Município de Assis – FEMA, como requisito parcial à obtenção do Certificado de Conclusão.}
			\p{\textbf{Orientando(a):} \orientando
			\\\textbf{Orientador(a):} \orientador}
			}
		}
		\vfill
		\textbf{Assis/SP\\\the\year}
	\end{titlepage}
}

% desenhando pagina com assinatura da banca
\def\paginaComissaoExaminadora{
	\begin{titlepage}
		\centering
		\includegraphics[keepaspectratio]{Imagens/fema}
		\jumpLine{7}
		\textbf{\orientando}
		\vfill
		\textbf{\titulo}
		
		\textoDireta{8cm}{\jumpLine{2}\footnotesize{
				\p{Trabalho de Conclusão de Curso apresentado ao Instituto Municipal de Ensino Superior de Assis, como requisito do Curso de Graduação, avaliado pela seguinte comissão examinadora:}
			}
		}
		\vfill
		\signature{Orientador}{\orientador}
		\signature{Examinador}{\examinador}
		\vfill
		\textbf{Assis/SP\\\the\year}
	\end{titlepage}
}

%pagina de resumo e abstract
\def\showresumo#1#2{
	\showpageabstract{Resumo}{#1}{Palavras-chave}{#2}
}
\def\showabstract#1#2{
	\showpageabstract{Abstract}{#1}{Keywords}{#2}
}

\def\showpageabstract#1#2#3#4{
\begin{titlepage}\centering{\large\textbf{#1}}
	\jumpLine{1}
	\p{#2}
	\jumpLine{2}
	\p{\textbf{#3:} #4}
\end{titlepage}
}

