\section{CONCLUSÕES}

\p{Neste trabalho é apresentado, o desenvolvimento e idealização de todo uma metodologia de Segurança para ser aplicado a uma aeronave não tripulada. Visando garantir a segurança foi desenvolvido um Diagrama de Mitigação de falha geral de uma aeronave, alem da implementação de um computador, capaz de garantir a segurança da aeronave. Tambem foi desenvolvido uma modificação no planejador de voo para oferecer a o operador uma visão geral da situação dos componentes da aeronave, e situação do sistema de segurança.}
\p{Toda o trabalho de desenvolvimento realizado, demonstra uma solução para garantir a segurança de uma aeronave não tripulada.}
\p{Entretanto, durante o desenvolvimento deste trabalho, tornou-se evidente que a elaboração de um sistema de segurança voltada para aeronave não tripulada, é altamente dependente do Tipo de Aeronave e sua aplicação. Isto torna muito mais complexo o desenvolvimento de um sistema flexível e universal, que se adeque a qualquer tipo de aeronave e as normas dos ogões reguladores.}
\p{Outro ponto muito importante identificado, foi que nem todas as informações necessárias para estabelecer a segurança na aeronave, podem ser obtidas através da Pixhawk, assim sendo necessario o desenvolvimento de uma placa especializada em coleta de dados. Para ser possivel coletar informações sobre: nível de combustível, consumo e velocidade de rotação dos motores, temperatura e desempenho de baterias, e acionamento do payload são cruciais para garantir a segurança e eficácia do voo.}
\p{A partir do desenvolvimento e da implementação deste sistema de segurança para VANTs, teremos um sistema de segurança em conformidade com as normas de segurança estabelecidas pelos órgãos reguladores competentes do território brasileiro, assim tornando a aeronave segura e homologável perante as exigências solicitadas pelos mesmo.}
\subsection{TRABALHOS FUTUROS}
\p{Como trabalho futuro, pretende-se desenvolver um hardware universal capaz de capturar dados de todos os dispositivos e sensores presentes na aeronave, a fim de obter o máximo de informações possível, pois o controlador da \citeauthor{pixhawk} não fornece muitas destas informações que são necessárias para realizar a verificação de funcionamento dos componentes presente na aeronave.}
\p{Portanto, desenvolver um hardware universal capaz de capturar essas informações é essencial para maximizar a eficiência e segurança das operações aéreas.}