\newpage\section{RESULTADO}
\p{Neste trabalho é apresentado, o desenvolvimento e idealização de todo uma metodologia de Segurança para ser aplicado a uma aeronave não tripulada. Visando garantir a segurança foi desenvolvido um Diagrama de Mitigação de falha geral de uma aeronave, alem da implementação de um computador, capaz de garantir a segurança da aeronave. Tambem foi desenvolvido uma modificação no Planejador de Voo para oferecer a o Operador uma visão geral da situação dos componentes da aeronave, e situação do Sistema de Segurança.}
\p{Toda o trabalho de desenvolvimento realizado, demonstra uma solução para garantir a segurança de uma aeronave não tripulada.}
\p{Entretanto, durante o desenvolvimento deste trabalho, tornou-se evidente que a elaboração de um Sistema de Segurança voltada para aeronave não tripulada, é altamente dependente do Tipo de Aeronave e sua aplicação. Isto torna muito mais complexo o desenvolvimento de um sistema flexível e universal, que se adeque a qualquer tipo de aeronave e as normas dos ogões reguladores.}
\p{Outro ponto muito importante identificado, foi que nem todas as informações necessárias para estabelecer a segurança na aeronave, podem ser obtidas através da Pixhawk, assim sendo necessario o desenvolvimento de uma placa especializada em coleta de dados. Para ser possivel coletar informações sobre: nível de combustível, consumo e velocidade de rotação dos motores, temperatura e desempenho de baterias, e acionamento do payload são cruciais para garantir a segurança e eficácia do voo.}